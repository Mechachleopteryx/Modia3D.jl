%%%%%%%%%%%%%%%%%%%%%%%
% Section 
%%%%%%%%%%%%%%%%%%%%%%%
\section{Introduction}

The open source Modia prototype platform, see table~\ref{table:ModiaPlatform}, consists of various packages
based on the Julia programming language \cite{bezanson2017julia} to model and simulate physical systems
described by differential and algebraic equations. The goal is
to simulate, e.g. robots, vehicles, aircraft, buildings or power plants,
and to experiment with novel features for the next Modelica\textsuperscript{\textcopyright} language
generation\footnote{\href{https://www.modelica.org/modelicalanguage}{https://www.modelica.org/modelicalanguage}}.

Modia3D\footnote{\href{https://github.com/ModiaSim/Modia3D.jl}{https://github.com/ModiaSim/Modia3D.jl}}
is an experimental modeling and simulation environment to provide 3D geometry and 3D mechanical systems
for the Modia platform. The goal is to fully integrate Modia3D with Modia's equation-based modeling and
provide a common graphical user interface with the web app Modiator. This will
allow to model, for example, the 3D-mechanics of a robot with Modia3D and the electrical motors, gear boxes,
sensors and controllers with Modia. One enhancement with respect to the widely-used Modelica language
will be that specialized algorithms for 3D kinematics and mechanics, combined with equation-based modeling,
allows more robust and efficient simulations of complex systems.

Modia3D uses ideas of modern computer
    game engines\footnote{\href{https://en.wikipedia.org/wiki/List_of_game_engines}{https://en.wikipedia.org/wiki/List\_of\_game\_engines}},
to achieve a highly flexible setup of mechanical systems including collision handling. 
Contrary to game engines, numerical integration is performed with a variable-step solver
(IDA via the Sundials.jl Julia package \cite{Sundials2005,Rackauckas2017}) as needed for applications
where system simulations have to match reality with a certain precision.
In Modia3D, collision handling is performed with elastic response calculation and not
with impulses, as it is common for game engines. The reason is that simulation results are closer to reality
and it is easier to treat complex contact situations correctly. For example, it is hard to model situations
with impulses in a physically correct way if combined effects such as rolling, sliding, friction and 
collisions with several bodies occur at the same time instant. 

Modia3D uses algorithms and features from multi-body
programs\footnote{\href{http://real.uwaterloo.ca/~mbody/\#Software}{http://real.uwaterloo.ca/mbody/\#Software}}
such as rigid mathematical formulation with hybrid DAEs (Differential Algebraic Equations), for example, to
describe closed kinematic loops, whereas game engines often make compromises here. On the other hand,
multi-body programs are usually far from the flexible setup of games (see also section~\ref{sect_flexibleDefinition}).

Modia3D provides a generic interface to visualize simulation results with different 3D renderers. Currently, the free community edition as well as the professional edition\footnote{\href{https://visualization.ltx.de/}{https://visualization.ltx.de/}} of the
\emph{DLR Visualization}  library\footnote{\href{http://www.systemcontrolinnovationlab.de/the-dlr-visualization-library/}{http://www.systemcontrolinnovationlab.de/the-dlr-visualization-library/}} \cite{bellmann2009, hellerer2014} are supported. Another team is developing Modiator, a free 2D/3D web-based authoring and rendering tool.

The user's view of Modia3D was introduced in \cite{Neumayr2018} showing the very flexible definition of 3D systems.
Some key algorithms are discussed in \cite{Neumayr2017,Neumayr2019}. Collision handling with elastic response calculation and error controlled integration
is challenging and this article discusses some of the difficulties and how they are solved. Further, it gives an overview of Modia3D from a user's perspective, and in particular how collisions between objects are defined.

\begin{table}[t]
	\tbl{Julia packages of the Modia platform \footnotesize{(\href{https://github.com/ModiaSim}{github.com/ModiaSim})}.}{
		\begin{tabular}{@{\hspace{2pt}}ll@{\hspace{0pt}}} \hline
			Modia      & Equation-based modeling of physical systems \\
			Modiator   & 2D and 3D web app model editor (\textit{not yet public}) \\
			ModiaMath  & Simulation environment for differential-algebraic equations \\
			Modia3D    & 3D geometries and 3D mechanics with collision handling \\
			ModiaMedia & Thermodynamic property models \\ 
			Modelia    & Modelica model importer (\textit{not yet public})
	\end{tabular}}
	\label{table:ModiaPlatform} 
\end{table}

%Furthermore, existing elastic response formulations are combined and enhanced so that a minimum number of material data of colliding objects has to be provided.
